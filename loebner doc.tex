
 \documentclass[14pt,a4paper]{report}
 
 \usepackage[pagebackref=false,colorlinks,linkcolor=blue,citecolor=magenta]{hyperref}
 
 \usepackage[nottoc]{tocbibind}
 \usepackage{calc}
 \usepackage{eso-pic}
 \usepackage{hyperref}
 \usepackage {fancybox}
 \usepackage{graphicx, xcolor}
 
 \usepackage{xepersian}
 \settextfont{B Nazanin}
 \title{نگاهی به برندگان جایزه  \lr{Loebner}}
\author{ آقایان محمدرضا محمدی، امیر محمدی و رضا رضایی}
\renewcommand{\bibname}{مراجع}
\begin{document}


	
 	\maketitle
 	\tableofcontents
 	
 	
 	\chapter{جایزه \lr{Loebner} چیه؟} 
 	هر ساله یک رقابتی در زمینه هوش مصنوعی برگزار میشود که در آن جوایزی به برنامه های کامپیوتری اهدا میشود که از نظر داوران،  شبیه ترینِ به انسان هاست. قالب این مسابقه ها بر اساس یک تست استاندارد تورینگ هست. در هر دور، یک قاضی انسانی همزمان مکالمه های متنی را با یک برنامه کامپیوتری و یک انسان از طریق کامپیوتر برگزار میکند، بر اساس پاسخ های دریافت شده، قاضی باید تصمیم بگیرد که کدام، کدام است!!!
 	
 	
 	
\end{document}
