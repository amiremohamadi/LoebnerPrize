
 \documentclass[14pt,a4paper]{report}
 
 \usepackage[pagebackref=false,colorlinks,linkcolor=blue,citecolor=magenta]{hyperref}
 
 \usepackage[nottoc]{tocbibind}
 \usepackage{calc}
 \usepackage{eso-pic}
 \usepackage{hyperref}
 \usepackage {fancybox}
 \usepackage{graphicx, xcolor}
 
 \usepackage{xepersian}
 \settextfont{B Nazanin}
 \title{نگاهی به برندگان جایزه  \lr{Loebner}}
\author{ آقایان محمدرضا محمدی، امیر محمدی و رضا رضایی}
\renewcommand{\bibname}{مراجع}
\begin{document}


	
 	\maketitle
 	\tableofcontents
 	
 	
 	\chapter{جایزه \lr{Loebner} چیه؟} 
 	هر ساله یک رقابتی در زمینه هوش مصنوعی برگزار میشود که در آن جوایزی به برنامه های کامپیوتری اهدا میشود که از نظر داوران،  شبیه ترینِ به انسان هاست. قالب این مسابقه ها بر اساس یک تست استاندارد تورینگ هست. در هر دور، یک قاضی انسانی همزمان مکالمه های متنی را با یک برنامه کامپیوتری و یک انسان از طریق کامپیوتر برگزار میکند، بر اساس پاسخ های دریافت شده، قاضی باید تصمیم بگیرد که کدام، کدام است!!!
 	
 	\section{\lr{Loebner} کیست؟}
 	هیو لوبنر متولد 26 مارس 1942 و وفات 4 دسامبر 2016 به عنوان حامی جایزه لوبنر و تجسم آزمون تورینگ مشهور بود. او یک مخترع آمریکایی دارای 6 اختراع و یک فعال اجتماعی آشکار برای جزم زدایی فحشا بود.
 	
 	
 	\chapter{برندگان جایزه \lr{loebner}}
 	\section{\lr{Steve Worswick}}
 	استیو وورسویک، سازنده برنامه \lr{Mitsuku} توانسته است در سال های 2013، 2016، 2017، 2018 و 2019 جایزه لوبنر را از آن خود کند. \lr{Mitsuku} یک ربات چت هست که از فن آوری \lr{AIML} استفاده میکند. \lr{Mitsuku}  به عنوان یک بازی فلش در بازی های \lr{Mousebreaker}  و در فیسبوک مسنجر، چت گروهی \lr{Telegram} \lr{Twitch} و \lr{Kik} با نام کاربری "\lr{Pandorabots}" موجود است و با همین نام در اسکایپ نیز در دسترس بود اما توسط توسعه دهنده آن حذف شد.
 	
 برنامه 	\lr{Mitsuku} ادعا میکند که یک زن 18 ساله از شهر لیدزِ انگلیس هست. این شامل کلیه پرونده های \lr{AIML} آلیس هست، با تعداد بسیاری از اضافات مکالمات ایحاد شده توسط کاربر و همیشه یک کار در حال انجام هست. \lr{Worswick} ادعا میکند که از سال 2005 در این زمینه کار کرده است. 
 
 سیبسبیسب d
 	
 	
 	
 
 	
\end{document}
